\documentclass[a4paper,oneside,11pt,leqno]{article}

\usepackage[spanish, activeacute]{babel}
\usepackage[utf8]{inputenc}
\usepackage{amsfonts}
\usepackage{amsmath}
\usepackage{fancyhdr}
\usepackage{epic}
\usepackage{eepic}
\usepackage{amssymb}
\usepackage{hyperref}
\usepackage{fancybox}


\textwidth = 16truecm
\textheight = 24truecm
\oddsidemargin =-20pt
\evensidemargin = 5pt
\topmargin=-1truecm

\begin{document}
%\thispagestyle{empty}
\hrule
\vskip 5pt

\noindent{\bf TRABAJO DE FIN DE GRADO EN MATEMÁTICAS} \\
Departamento de Matemáticas\\
Universidad Autónoma de Madrid\\
\noindent ({\it Curso académico 2020-21})
\vskip 6pt \hrule

\vskip 1mm

\noindent{\bf Título del proyecto}: Análisis de datos funcionales: aproximación al problema de clasificación.

\vskip 1mm

\noindent{\bf Nombre y apellidos}: Pedro Martín Rodríguez-Ponga Eyriès

\vskip 1mm

\noindent{\bf Nombre de los tutores}: Luis Alberto Rodríguez Ramírez y José Luis Torrecilla Noguerales

\vskip 2mm

\centerline{\bf INFORME INTERMEDIO}

\vskip 2mm

\begin{enumerate}

      \item[1.-] {\bf Labor desarrollada hasta la fecha}:

            En el mes de septiembre, comenzaron las reuniones con los tutores que han sido o bien semanales o bien quincenales.
            Las reuniones se han enfocado principalmente en lo siguiente:
            \begin{itemize}
                  \item Introducción al análisis de datos funcionales.
                  \item Recomendación de bibliografía.
                  \item Resolución de dudas surgidas en la lectura.
                  \item Verificar la asimilación de conceptos.
                  \item Proporcionar una idea más intuitiva de algunos temas ilustrándolos mediante ejemplos más concretos.
            \end{itemize}

            Por otro lado, se ha mantenido una reunión adicional todas las semanas para tratar el proyecto paralelo de ampliación
            de una librería de \textbf{Python} \cite{scikit-fda} de análisis de datos funcionales.

      \item[2.-] {\bf Esquema de los distintos apartados del trabajo}:

            Una estructura orientativa del trabajo podría ser:

            \begin{itemize}
                  \item 0. Introducción al análisis de datos funcionales \cite{Ramsay-Silverman,Ferraty-Vieu,Cuevas}.
                        \subitem 0.1 Definición y motivación.
                        \subitem 0.2 Dificultades.
                  \item 1. Introducción al problema de clasificación \cite{tesis,MFC}.
                        \subitem 1.1 Clasificador de Bayes: definición y optimalidad.
                        \subitem 1.2 Clasificación con datos funcionales.
                  \item 2. Clasificación basada en profundidades \cite{depth-classification}.
                        \subitem 2.1 Definición y propiedades básicas.
                        \subitem 2.2 Profundidades en $\mathbb{R}$, $\mathbb{R}^d$, espacios funcionales.
                        \subitem 2.3 Clasificadores basados en profundidades \cite{maximum-depth,DD-classifier,DDG-classifier}.
                        \subitem 2.4 Comparativa y simulación.
                  \item 3. Clasificación basada en distancias.
                        \subitem 3.1 Definición de métrica y semi-métrica.
                        \subitem 3.2 Mahalanobis funcional.
                        \subitem 3.3 Clasificadores basados en distancias \cite{trimmed-means}.
                        \subitem 3.4 Comparativa y simulación.
                  \item 4. Regresión logística.
            \end{itemize}

      \item[3.-] {\bf Descripción del proyecto}:

            La motivación del proyecto es el estudio de la extensión del problema de clasificación supervisada a
            un espacio funcional, que puede tener infinitas dimensiones. Esta extensión permite trabajar con nuevos tipos de
            datos, pero también añade nuevas dificultades como, por ejemplo, que las normas no son equivalentes. Concretamente,
            parte del trabajo se centrará en el estudio de cómo la elección de una métrica (o semi-métrica) afecta
            a los resultados de clasificación.

            Entre los principales resultados, se espera realizar una simulación ilustrando los diferentes
            métodos de clasificación, sus similitudes y sus diferencias.

            Además, los diferentes métodos se implementarán en \textbf{Python} para poder aplicarlos a conjuntos de datos meteorológicos,
            médicos, poblacionales...

      \item[4.-] {\bf Grado de consecución de los objetivos y plan de trabajo a desarrollar en la segunda mitad del periodo}:

            En la primera parte del curso, se ha realizado un estudio de las principales referencias \cite{Ramsay-Silverman, Ferraty-Vieu},
            centrándose en la teoría general del análisis de datos funcionales, así como en el problema de clasificación supervisada.
            Hasta la fecha, el mayor esfuerzo ha sido en métodos de clasificación basados en profundidades.

            Por otro lado, se ha estudiado análisis funcional de cara a aprovechar algunas propiedades de espacios funcionales dotados
            de un producto escalar como puede ser el espacio de Hilbert $L^2$ \cite{papa-Rudin}.

            En la segunda mitad del periodo, se pretende centrarse en clasificación basada en distancias insistiendo en
            la distancia de Mahalanobis. En caso de permitirlo el tiempo, se finalizaría con la regresión logística.

            Respecto de las reuniones, se continuará fundamentalmente con el mismo plan de trabajo.

            \begin{thebibliography}{10}

                  \bibitem{Ramsay-Silverman}
                  \textsc{Ramsay, J. and Silverman, B. W.} (2005.) Functional Data Analysis.
                  Springer Series in Statistics. Springer.

                  \bibitem{Ferraty-Vieu}
                  \textsc{Ferraty, F. and Vieu, P.} (2006.) Nonparametric Functional Data Analysis.
                  Springer Series in Statistics. Springer.

                  \bibitem{tesis}
                  \textsc{Torrecilla, J. L.} (2015.) On the Theory and Practice of Variable
                  Selection for Functional Data [Doctoral dissertation]. Universidad Autónoma de Madrid, España.

                  \bibitem{Cuevas}
                  \textsc{Cuevas, A.} (2013). A partial overview of the theory of statistics with functional data.
                  Journal of Statistical Planning and Inference. \url{http://dx.doi.org/10.1016/j.jspi.2013.04.002}

                  \bibitem{scikit-fda}
                  \textsc{GAA-UAM.} (2021). scikit-fda. \url{https://github.com/GAA-UAM/scikit-fda}.

                  \bibitem{MFC}
                  \textsc{Hubert, M., Rousseeuw, P. J., and Segaert, P.} (2017).
                  Multivariate and functional classification using depth and distance.
                  Advances in Data Analysis and Classification. 11. 445–466. 10.1007/s11634-016-0269-3.

                  \bibitem{trimmed-means}
                  \textsc{Fraiman, R. and Muniz, G.} (2001). Trimmed means for functional
                  data. Test, 10, 419-440.

                  \bibitem{depth-classification}
                  \textsc{Pintado, S., and Romo, J.} (2009). On the Concept of Depth for Functional Data.
                  Journal of the American Statistical Association. 104. 10.1198/jasa.2009.0108.

                  \bibitem{maximum-depth}
                  \textsc{Ghosh, A. K. and Chaudhuri, P.} (2005b). On maximum depth and
                  related classifiers. Scandinavian Journal of Statistics, 32, 327–350.

                  \bibitem{DD-classifier}
                  \textsc{Li, J., Cuesta-Albertos, J. A., and Liu, R. Y.} (2012).
                  DD-classifier: Nonparametric classification procedure based on DD-plot.
                  Journal of the American Statistical Association, 107(498):737-753.

                  \bibitem{DDG-classifier}
                  \textsc{Cuesta-Albertos, J. A., Febrero-Bande, M. and Oviedo de la Fuente, M.}
                  (2017). The DDG-classifier in the functional setting. TEST, 26. 119-142.

                  \bibitem{papa-Rudin}
                  \textsc{Rudin, W.} (1974).
                  Real and complex analysis, 2nd ed. McGraw-Hill, Inc., USA.

            \end{thebibliography}


\end{enumerate}

\end{document}

