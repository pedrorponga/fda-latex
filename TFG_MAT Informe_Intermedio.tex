\documentclass[a4paper,oneside,11pt,leqno]{article}

\usepackage[latin1]{inputenc}
\usepackage[spanish]{babel}
\usepackage{amsfonts}
\usepackage{amsmath}
\usepackage{fancyhdr}
\usepackage{epic}
\usepackage{eepic}
\usepackage{amssymb}
\usepackage{hyperref}
\usepackage{fancybox}


\textwidth = 16truecm 
\textheight = 24truecm
\oddsidemargin =-20pt
\evensidemargin = 5pt
\topmargin=-1truecm

\begin{document}
%\thispagestyle{empty}
\hrule
\vskip 6pt

\noindent{\bf TRABAJO DE FIN DE GRADO EN MATEM\'ATICAS} \\
Departamento de Matem\'aticas\\
Universidad Aut\'onoma de Madrid\\
\noindent ({\it Curso acad\'emico 2020-21}) 
\vskip 6pt \hrule

\vskip 5mm

\noindent{\bf T\'itulo del proyecto}:

\vskip 5mm

\noindent{\bf Nombre y Apellidos}:

\vskip 5mm

\noindent{\bf Nombre del tutor(es)}:

\vskip 2cm


\centerline{\bf INFORME INTERMEDIO \footnote{ El informe debe ser elaborado por el estudiante y presentado al tutor -o tutores- que deber\'a dar su conformidad antes de ser entregado al coordinador.}}

\vskip 5mm

\begin{enumerate}

\item[1.-] {\bf Labor desarrollada hasta la fecha}: (\textit{reuniones con el tutor; b\'usqueda de bibliograf\'ia; planteamiento de los objetivos.})

\item[2.-] {\bf Esquema de los distintos apartados del trabajo}: (\textit{puede usarse como gu\'ia la propia tabla de contenidos.})

\item[3.-] {\bf Descripci\'on del proyecto}: (\textit{motivaci\'on; principales resultados y, en su caso, aplicaciones que se esperan obtener.}) M\'aximo 2 p\'aginas.

\item[4.-] {\bf Grado de consecuci\'on de los objetivos y plan de trabajo a desarrollar en la segunda mitad del periodo}: 

\item[5.-] {\bf Bibliograf\'ia usada hasta la fecha o que se piensa utilizar}: \\
Ejemplo:

\begin{thebibliography}{10}

\bibitem{Abel} 
    \textsc{Abel, N.\,H.}: 
    Beweis eines Ausdrucks, von welchem die Binomial-Formel ein einzelner Fall ist. 
    \textit{J. Reine angew. Math.} {\bf1} (1826), 159--160.

\end{thebibliography}


\end{enumerate}

\end{document}

